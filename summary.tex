\documentclass{scrartcl}
\usepackage{setspace}
\begin{document}
\onehalfspacing
\obeylines
\setlength{\parindent}{0pt}
\begin{center}\LARGE\textbf{Technology and Innovation Management: Introduction}\end{center}

\section*{Introduction}
\subsection*{Why Does Innovation Matter?}
- main driver of growth
- OECD estimate is 60-70\% of labour productivity growth
- under capitalism innovation becomes mandatory
\textbf{Product or service innovation:}
- to generate/increase Sales
- must be established on the market
\textbf{Process innovation:}
- to enable/improve production of goods or services
- must be established inside the organisation
\textbf{Business model innovation}
- Reshuffling of value proposition, processes, products, services, \dots
- Example: Ikea selling unassembled furniture
\subsection*{The Vasa}
- Failed Innovation
- Lack of communication, nobody dares to voice concerns

\section*{Patterns in Innovative Activity}
\subsection*{Long-term patterns of technological change}
- Technological change is cumulative and evolutionary
- Most innovations are new combinations of existing technologies or the introduction of new elements into existing systems
- some changes can be identified as revolutionary
\subsection*{Innovation shifts socio-economic paradigms}
K-waves/Schumpeter's waves: Long cycles of economic growth and decline (50 years), where new inventions start new cycles of growth

\section*{Patterns in Technological Evolution}
\subsection*{S-curve}
- X-axis: Aggregate R\&D spending (or time)
- Y-axis: Performance over time
1. Emergence: low performance
2. Rapid improvement: accelerating performance
3. Declining improvement: deccelerating performance
4. Maturity: saturated performance
- Often, a technology follows the S-Curve
- New technology at some point surpasses old technology's s-curve
- S-curve does not always represent reality (e.g. lithography)
\subsection*{Sailing ship phenomenon}
- S-cuves of sailing ships and then steam ships
- Right before steam ships overtake sailing ships in performance, new better sailing ship technology (the Cutty Sark)
- Final sprint of old technology motivated by accelerating performance of new technology
\subsection*{Product Life Cycle (PLC)}
Fluid Phase:
- In the early phase of a new product, frequent product changes occur
- diverse design
- Unspecified focus of R\&D
- entrepreneurial organisation
- Much competition, more players enter market than exit
Transitional Phase:
- Major process changes
- One product design
- One R\&D focus
- Organisation through project and task groups
Specific Phase:
- Incremental changes and innovations
- Only standard products
- R\&D on incremental product technologies
- Well structured organisation
- More players leave market than enter
\subsection*{Adopter Categories}
- Innovators
- Early adopters
- Early majority
- Late majority
- Laggards
Jeffrey Moore: Crossing the Chasm:
- Chasm is hurdle between early adopters and early majority
- Central question in marketing of new technology

\section*{Who Innovates, and Why?}
Costs and benefits of innovation are the dominant drives of innovative activity
- Whoever gains the most is most likely to perform it
- Whoever has the lowest cost is most likely to do it
Schumpeters classical question: What market structure is most conducive to innovation?
- Schumpeter I: Entrepreneurs and new firms drive innovation $=>$ fragmented markets
- Schumpeter II: Large firms drive innovation $=>$ markets with some monopoly power
\subsection*{Arrow's Model}
What price would the innovator be willing to pay for the innovation?
- $q$: Quantity
- $p$: Price
- $p(q) = 100 - q$: Linear demand curve. Higher price causes less demand (reversed linear function).
- $c_0$: Cost before innovation
- $c_1$: Cost after innovation
- $TR = q \times p$: Total revenue if selling $q$ products at price $p$.
- $MR$: Marginal Revenue when selling one more product at price $p$. Ideally, this should be equal to marginal cost $c_0$, to maximize the potential profit. We can find the marginal revenue by substituting the price by the demand function in the total revenue formula, and then finding the zero point by deriving by $q$.
$$TR = q \times (100 - q)$$
$$TR = 100q - q^2$$
$$MR = \frac{\partial TR}{\partial q} = 100 - 2q$$
The point where the resulting line from $MR = 100 - 2q$ intersects $c_1$ tells us how much how much quantity $q_1$ we should sell to maximize profits.
Definition radical innovation:
New Monopoly Price $p_1$ $<$ Old Marginal Cost $c_0$
\subsubsection*{Case A: Ex-ante perfect competition $=>$ ex-post monopoly:}
Profit ex-ante: $0$
Profit ex-post: $area(ABCD) = D - A \cdot C - D = (p_1 - c_1) * q_1$
- $A = (0, c_1)$
- $B = (q_1, c_1)$
- $C = (q_1, p(q_1))$
- $D = (0, p(q_1))$
\subsubsection*{Case B: Ex-ante monopoly $=>$ ex-post monopoly}
Profit ex-ante: $area(EFGH)$. Calculation is the same as for $area(ABCD)$ but with $c_0$ instead of $c_1$.
Profit ex-post: $area(ABCD)$
\subsubsection*{Case C: Ex-ante social planner, ex-post social planner}
Profit ex-ante: $0$, because social planner sets $p = c_0$
Profit ex-post: $0$, because $p = c_1$
Conclusion: Incentive to invest is lower for private companies than for social planners, because potential profit is always less than social incentive.
Welfare ex-ante: $area(EJK)$
Welfare ex-post: $area(AIK)$
which fills the whole are under the demand curve, which means the social planner gains more from the innovation.
This means case A, the ex-ante monopolist, has the lowest incentive due to replacement effect. Followed by case B, the firm in ex-ante perfect competition. The Social planner has the highest incentive.
\subsection*{Market Structure}
- Efficiency Effect: The monopolist is willing to pay more for the innovation (than a entrant) to remain the monopolist even in case of a non-drastic innovation.
- Cannibalization Effect: By innovating (drastic innovation) the existing firm would lose profits from exisiting product.
Empirical studies confirm that drastic innovation often comes from new entrants, not from incumbents.
\subsection*{Advantages of newness}
\textbf{"Liability of newnewss" (increased failure rate vs older firms):}
- roles and tasks need to be assigned
- lack reputation
- lack experience
- no relationships
- have to rely on strangers
- limited resources
- low variety of skill in the firm, some critical skills are lacking
- no buffer for crisis
- disadvantage in the job market
- low market power
- no organizational slack (resources beyond current need)
\textbf{Advantages of new firms:}
- can pursue completly new approaches
- more lean structure
- flexible and open culture
- can hire people exactly for task (no re-training)
- more younger and entrepreneurial employees
- more flexible processes
- company structure easier to identify
- direct communication
- fast decision-making
- job satisfaction higher
\subsection*{Disruptive Innovation}
In some industries big firms bring most innovation, both incremental and radical ones:
- These innovations are technologically straightforward
- Initially dont satisfy customers in established markets, sold to niche or new markets
- Performance of new technology grows
- New tech supplants old tech even in established markets
Disruptive innovation is difficult for market-related reasons. Incumbents also focus too much on existing customers.

\section*{Innovation Strategy}
Strategy is determination of long-term goals of a business as well as suitable steps and the allocation of resources to attain them. It intends to establish a long-term competitive advantage.
Change of strategy is typically very costly and not possible in the short run.
\subsection*{Strategy Development}
\textbf{Internal/Resource based-View}
- Strengths
- Weaknesses
\textbf{External/Market based-View}
- Opportunities
- Threats
\textbf{Porters Five Forces}
- Threat of new entrants
- Threat of substitute products
- Bargaining Power of Suppliers
- Bargaining Power of Buyers
- Existing Industry Rivalry
\textbf{Potential (Re-)Sources of Competitive Advantage}
- Valuable
- Rare
Sustainable Competitve Advantage:
- Inimitable
- Non-substitutable
Resources/capabilities are difficult to imitate if they are:
- Tacit (Tacit knowledge = difficult to transfer written/verbally)
- Path dependent
- Socially complex
- Casually ambiguous
- Protected by effective IP
- Unique in their nature (e.g. location)
\textbf{Resources for Innovation}
- Financial resources
- Human resources
- Technological resources
- Marketing resources
- Organizational resources
- Networking resources
\subsection*{Key Pillars of Successful Innovation Strategy}
\textbf{Objectives}
An innovation strategy defines specific and precise innovation objectives in line with long-term goals.
\textbf{Areas of Focus}
Innovation strategy defines whats in-bounds and out-of-bounds.
\textbf{Type of innovation and accepted risk}
Three positions under uncertainty:
- Taking big bets (aiming for radical innovation)
- Hedging bets (diversifying, incremental innovation)
- Wait and see (imitate later)
\textbf{Budget and Resources}
The allocation of resources to the various strategic or project types ensures strategic alignment.
\textbf{R\&D Organization}
E.g. Decentralization of R\&D efforts based on local expertise.
\subsection*{Types of Innovation Strategies}
- Proactive
- Active
- Reactive
- Passive
\subsection*{Technology Push vs Market Pull}
Market Pull:
- Look for market needs
- Then look for technology to fulfill these needs
Technology Push:
- Look for new technologies with potential
- Question of needs comes second
...Or combination
Problems with Technology Push:
- Focus on easy problems
- Narrow focus on own solution
- Mismatch between own ideas and corporate interest
Problems with Market Pull:
- Focus on most easily identified needs but only minor potential
- Too much focus on a particular application
- Locked into present products, only incremental improvements




\end{document}
