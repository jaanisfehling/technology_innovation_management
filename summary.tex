\documentclass{scrartcl}
\usepackage{setspace}
\begin{document}
\onehalfspacing
\obeylines
\setlength{\parindent}{0pt}
\begin{center}\LARGE\textbf{Technology and Innovation Management: Introduction}\end{center}

\section*{Introduction}
\subsection*{Why Does Innovation Matter?}
- main driver of growth
- OECD estimate is 60-70\% of labour productivity growth
- under capitalism innovation becomes mandatory
\textbf{Product or service innovation:}
- to generate/increase Sales
- must be established on the market
\textbf{Process innovation:}
- to enable/improve production of goods or services
- must be established inside the organisation
\textbf{Business model innovation}
- Reshuffling of value proposition, processes, products, services, \dots
- Example: Ikea selling unassembled furniture
\subsection*{The Vasa}
- Failed Innovation
- Lack of communication, nobody dares to voice concerns

\section*{Patterns in Innovative Activity}
\subsection*{Long-term patterns of technological change}
- Technological change is cumulative and evolutionary
- Most innovations are new combinations of existing technologies or the introduction of new elements into existing systems
- some changes can be identified as revolutionary
\subsection*{Innovation shifts socio-economic paradigms}
K-waves/Schumpeter's waves: Long cycles of economic growth and decline (50 years), where new inventions start new cycles of growth

\section*{Patterns in Technological Evolution}
\subsection*{S-curve}
- X-axis: Aggregate R\&D spending (or time)
- Y-axis: Performance over time
1. Emergence: low performance
2. Rapid improvement: accelerating performance
3. Declining improvement: deccelerating performance
4. Maturity: saturated performance
- Often, a technology follows the S-Curve
- New technology at some point surpasses old technology's s-curve
- S-curve does not always represent reality (e.g. lithography)
\subsection*{Sailing ship phenomenon}
- S-cuves of sailing ships and then steam ships
- Right before steam ships overtake sailing ships in performance, new better sailing ship technology (the Cutty Sark)
- Final sprint of old technology motivated by accelerating performance of new technology
\subsection*{Product Life Cycle (PLC)}
Fluid Phase:
- In the early phase of a new product, frequent product changes occur
- diverse design
- Unspecified focus of R\&D
- entrepreneurial organisation
- Much competition, more players enter market than exit
Transitional Phase:
- Major process changes
- One product design
- One R\&D focus
- Organisation through project and task groups
Specific Phase:
- Incremental changes and innovations
- Only standard products
- R\&D on incremental product technologies
- Well structured organisation
- More players leave market than enter
\subsection*{Adopter Categories}
- Innovators
- Early adopters
- Early majority
- Late majority
- Laggards
Jeffrey Moore: Crossing the Chasm:
- Chasm is hurdle between early adopters and early majority
- Central question in marketing of new technology

\section*{Who Innovates, and Why?}
Costs and benefits of innovation are the dominant drives of innovative activity
- Whoever gains the most is most likely to perform it
- Whoever has the lowest cost is most likely to do it
Schumpeters classical question: What market structure is most conducive to innovation?
- Schumpeter I: Entrepreneurs and new firms drive innovation => fragmented markets
- Schumpeter II: Large firms drive innovation => markets with some monopoly power
\subsection*{Arrow's Model}
Linear demand curve. Constant cost $c_0$. New innovation reduces cost to $c_1$.
Question: What price would the innovator pay for the innovation?
Ex-ante perfect competition => ex-post monopoly because 



\end{document}
