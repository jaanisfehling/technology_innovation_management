\documentclass{scrartcl}
\usepackage{setspace}
\begin{document}
\onehalfspacing
\obeylines
\setlength{\parindent}{0pt}
\begin{center}\LARGE\textbf{Technology and Innovation Management: Introduction}\end{center}

\section*{Introduction}
\subsection*{Why Does Innovation Matter?}
- main driver of growth
- OECD estimate is 60-70\% of labour productivity growth
- under capitalism innovation becomes mandatory
\textbf{Product or service innovation:}
- to generate/increase Sales
- must be established on the market
\textbf{Process innovation:}
- to enable/improve production of goods or services
- must be established inside the organisation
\textbf{Business model innovation}
- Reshuffling of value proposition, processes, products, services, \dots
- Example: Ikea selling unassembled furniture
\subsection*{The Vasa}
- Failed Innovation
- Lack of communication, nobody dares to voice concerns

\section*{Patterns in Innovative Activity}
\subsection*{Long-term patterns of technological change}
- Technological change is cumulative and evolutionary
- Most innovations are new combinations of existing technologies or the introduction of new elements into existing systems
- some changes can be identified as revolutionary
\subsection*{Innovation shifts socio-economic paradigms}
K-waves/Schumpeter's waves: Long cycles of economic growth and decline (50 years), where new inventions start new cycles of growth

\section*{Patterns in Technological Evolution}
\subsection*{S-curve}
- X-axis: Aggregate R\&D spending (or time)
- Y-axis: Performance over time
1. Emergence: low performance
2. Rapid improvement: accelerating performance
3. Declining improvement: deccelerating performance
4. Maturity: saturated performance
- Often, a technology follows the S-Curve
- New technology at some point surpasses old technology's s-curve
- S-curve does not always represent reality (e.g. lithography)
\subsection*{Sailing ship phenomenon}
- S-cuves of sailing ships and then steam ships
- Right before steam ships overtake sailing ships in performance, new better sailing ship technology (the Cutty Sark)
- Final sprint of old technology motivated by accelerating performance of new technology
\subsection*{Product Life Cycle (PLC)}
Fluid Phase:
- In the early phase of a new product, frequent product changes occur
- diverse design
- Unspecified focus of R\&D
- entrepreneurial organisation
- Much competition, more players enter market than exit
Transitional Phase:
- Major process changes
- One product design
- One R\&D focus
- Organisation through project and task groups
Specific Phase:
- Incremental changes and innovations
- Only standard products
- R\&D on incremental product technologies
- Well structured organisation
- More players leave market than enter
\subsection*{Adopter Categories}
- Innovators
- Early adopters
- Early majority
- Late majority
- Laggards
Jeffrey Moore: Crossing the Chasm:
- Chasm is hurdle between early adopters and early majority
- Central question in marketing of new technology

\section*{Who Innovates, and Why?}
Costs and benefits of innovation are the dominant drives of innovative activity
- Whoever gains the most is most likely to perform it
- Whoever has the lowest cost is most likely to do it
Schumpeters classical question: What market structure is most conducive to innovation?
- Schumpeter I: Entrepreneurs and new firms drive innovation $=>$ fragmented markets
- Schumpeter II: Large firms drive innovation $=>$ markets with some monopoly power
\subsection*{Arrow's Model}
What price would the innovator be willing to pay for the innovation?
- $q$: Quantity
- $p$: Price
- $p(q) = 100 - q$: Linear demand curve. Higher price causes less demand (reversed linear function).
- $c_0$: Cost before innovation
- $c_1$: Cost after innovation
- $TR = q \times p$: Total revenue if selling $q$ products at price $p$.
- $MR$: Marginal Revenue when selling one more product at price $p$. Ideally, this should be equal to marginal cost $c_0$, to maximize the potential profit. We can find the marginal revenue by substituting the price by the demand function in the total revenue formula, and then finding the zero point by deriving by $q$.
$$TR = q \times (100 - q)$$
$$TR = 100q - q^2$$
$$MR = \frac{\partial TR}{\partial q} = 100 - 2q$$
The point where the resulting line from $MR = 100 - 2q$ intersects $c_1$ tells us how much how much quantity $q_1$ we should sell to maximize profits.
Definition radical innovation:
New Monopoly Price $p_1$ $<$ Old Marginal Cost $c_0$
\subsubsection*{Case A: Ex-ante perfect competition $=>$ ex-post monopoly:}
Profit ex-ante: $0$
Profit ex-post: $area(ABCD) = D - A \cdot C - D = (p_1 - c_1) * q_1$
- $A = (0, c_1)$
- $B = (q_1, c_1)$
- $C = (q_1, p(q_1))$
- $D = (0, p(q_1))$
\subsubsection*{Case B: Ex-ante monopoly $=>$ ex-post monopoly}
Profit ex-ante: $area(EFGH)$. Calculation is the same as for $area(ABCD)$ but with $c_0$ instead of $c_1$.
Profit ex-post: $area(ABCD)$
\subsubsection*{Case C: Ex-ante social planner, ex-post social planner}
Profit ex-ante: $0$, because social planner sets $p = c_0$
Profit ex-post: $0$, because $p = c_1$
Conclusion: Incentive to invest is lower for private companies than for social planners, because potential profit is always less than social incentive.
Welfare ex-ante: $area(EJK)$
Welfare ex-post: $area(AIK)$
which fills the whole are under the demand curve, which means the social planner gains more from the innovation.
This means case A, the ex-ante monopolist, has the lowest incentive due to replacement effect. Followed by case B, the firm in ex-ante perfect competition. The Social planner has the highest incentive.
\subsection*{Market Structure}
- Efficiency Effect: The monopolist is willing to pay more for the innovation (than a entrant) to remain the monopolist even in case of a non-drastic innovation.
- Cannibalization Effect: By innovating (drastic innovation) the existing firm would lose profits from exisiting product.
Empirical studies confirm that drastic innovation often comes from new entrants, not from incumbents.
\subsection*{Advantages of newness}
\textbf{"Liability of newnewss" (increased failure rate vs older firms):}
- roles and tasks need to be assigned
- lack reputation
- lack experience
- no relationships
- have to rely on strangers
- limited resources
- low variety of skill in the firm, some critical skills are lacking
- no buffer for crisis
- disadvantage in the job market
- low market power
- no organizational slack (resources beyond current need)
\textbf{Advantages of new firms:}
- can pursue completly new approaches
- more lean structure
- flexible and open culture
- can hire people exactly for task (no re-training)
- more younger and entrepreneurial employees
- more flexible processes
- company structure easier to identify
- direct communication
- fast decision-making
- job satisfaction higher
\subsection*{Disruptive Innovation}
In some industries big firms bring most innovation, both incremental and radical ones:
- These innovations are technologically straightforward
- Initially dont satisfy customers in established markets, sold to niche or new markets
- Performance of new technology grows
- New tech supplants old tech even in established markets
Disruptive innovation is difficult for market-related reasons. Incumbents also focus too much on existing customers.

\section*{Innovation Strategy}
Strategy is determination of long-term goals of a business as well as suitable steps and the allocation of resources to attain them. It intends to establish a long-term competitive advantage.
Change of strategy is typically very costly and not possible in the short run.
\subsection*{Strategy Development}
\textbf{Internal/Resource based-View}
- Strengths
- Weaknesses
\textbf{External/Market based-View}
- Opportunities
- Threats
\textbf{Porters Five Forces}
- Threat of new entrants
- Threat of substitute products
- Bargaining Power of Suppliers
- Bargaining Power of Buyers
- Existing Industry Rivalry
\textbf{Potential (Re-)Sources of Competitive Advantage}
- Valuable
- Rare
Sustainable Competitve Advantage:
- Inimitable
- Non-substitutable
Resources/capabilities are difficult to imitate if they are:
- Tacit (Tacit knowledge = difficult to transfer written/verbally)
- Path dependent
- Socially complex
- Casually ambiguous
- Protected by effective IP
- Unique in their nature (e.g. location)
\textbf{Resources for Innovation}
- Financial resources
- Human resources
- Technological resources
- Marketing resources
- Organizational resources
- Networking resources
\subsection*{Key Pillars of Successful Innovation Strategy}
\textbf{Objectives}
An innovation strategy defines specific and precise innovation objectives in line with long-term goals.
\textbf{Areas of Focus}
Innovation strategy defines whats in-bounds and out-of-bounds.
\textbf{Type of innovation and accepted risk}
Three positions under uncertainty:
- Taking big bets (aiming for radical innovation)
- Hedging bets (diversifying, incremental innovation)
- Wait and see (imitate later)
\textbf{Budget and Resources}
The allocation of resources to the various strategic or project types ensures strategic alignment.
\textbf{R\&D Organization}
E.g. Decentralization of R\&D efforts based on local expertise.
\subsection*{Types of Innovation Strategies}
- Proactive
- Active
- Reactive
- Passive
\subsection*{Technology Push vs Market Pull}
Market Pull:
- Look for market needs
- Then look for technology to fulfill these needs
Technology Push:
- Look for new technologies with potential
- Question of needs comes second
...Or combination
Problems with Technology Push:
- Focus on easy problems
- Narrow focus on own solution
- Mismatch between own ideas and corporate interest
Problems with Market Pull:
- Focus on most easily identified needs but only minor potential
- Too much focus on a particular application
- Locked into present products, only incremental improvements

\section*{Profiting from Innovation}
\subsection*{Who Benefits from Innovation}
\textbf{Customer Benefits}
In Arrows model, everything under the demand curve, above the new monopolists price is "Consumer surplus".
\subsection*{Conditions for Profitable Innovation}
- Appropriability regime
- Life cycle phase
- Complementary Assets
\subsection*{Appropriability Regime}
Denotes all factors that influence the possibility of profitable imitation of an innovation.
- Legal Protection
- Viability of Secrecy
- Characteristics of innovation/technology
Further Factors:
- Network effect
- Switching costs
- Scale effects
- ease of market entry
\subsection*{Life Cycle Phase}
- When number of product innovations peak, dominant design is established
- Product innovations decline afterwards
- After dominant design is established, number of process innovations increases
Pre-paradigmatic design phase (before dominant design was established):
- Innovators success depends on their ability to make their technology the dominant design
Paradigmatic design phase (after dominant design was established):
- Innovators success depends on the control of the complementary assets
\subsection*{Complementary Assets}
- Competitive manufacturing
- Distribution
- Service
- Complementary technologies
\textbf{Examples}
Specialized assets (unilateral dependence of asset on innovation or vice-versa):
- Dedicated service teams
- Specialized distribution channels
Co-specialized assets (bilateral dependence):
- Smartphone app eco-system
- Electric car charging stations
Generic assets (not specialized):
- Office space
- Off-the-shelf software
\subsection*{Combination of AR (Appropriability Regime) and CA (Complementary Assets)}
- Strong AR + Strong CA: Innovator and owner of CA profit
- Strong AR + Weak CA: Innovator profits
- Weak AR + Strong CA: Owner of CA profits
- Weak AR + Weak CA: Customers profit
\subsection*{Contract or Integrate?}
Contract:
- Lower capital needs
- Dependence on partner
- Imitation becomes easier
Integrate:
- Make imitation more difficult
- Capital intensive
- Time consuming
\subsection*{Cooperate?}
Benefits:
- Share potential benefit with other Firms, less competition
- Allow new firms to build on exisiting firms competencies
Risks:
- Willingness to pay for idea - but implies buyer knows aboout idea already
- Need for IP Rights

\section*{Protecting Intellectual Property}
\subsection*{Why care about IP Rights (IPR)}
- Help appropriate profits from innovation
- Increases incentive to innovate
- IPR of other firms can make operation/innovation difficult
- IPR contain valuable information
\subsection*{Formal vs Informal Information Protection}
Formal:
- Patents
- Registered Designs
- Trademarks
- Copyrights
- Confidentiality Agreements
Informal:
- Trade Secrets
- Lead time or first mover advantage
- Complexity of design
- Switching cost
- Network externalities
\subsection*{Patents}
- The main form of IP protection "Technological Invention"
- Right of ownership over an invention, granted by a government
- Patents are territorial rights; a German patent only relates to Germany
Whats patentable:
- Has to be novel
- Contain an "inventive step" compared to whats already known, something not obvious
- Be capable of industry application
Whats not patentable:
- Purely scientific discovery (without industry application)
- Scientific or mathematical method
- Asthetic creation like art or literature
- A device contrary to accepted physical laws
Special cases that cannot be patented in Germany and Europe:
- Computer program without physical effect
- Business method
- Invention of new animal or plant variety
- Treatment or Diagnosis of humans or animals
\subsubsection*{Patent Application}
- Describes invention
- Defines area of protection
- Application is made public after 18 months (1.5 years)
- Average time from application to decision: 4 years
- Patents last 20 years (25 for pharmaceutical)
- Have to be renewed (fee)
\subsubsection*{Benefits and Costs}
Benefits:
- Entry barries for rivals
- Profits from Licensing or cross-licensing (patent exchange)
- Image/signaling, especially for small firms seeking capital
Costs:
- Process costs, including attorney, research, fees (renewal fees), translation
- Invention is made public
- Detection of infringement
- Patent Assertion
\subsubsection*{Why do Firms Use Patents?}
- Prevent copying
- Blocking
- Prevent suits
- Enhance reputation
- For use in negotiations
- Licensings revenue
- Measure performance
\subsubsection*{Societal Benefits and Cost of Patents}
- Patents increase innovation because the innovator is "guaranted" to benefit from the innovation.
- Low use of innovation, which leads to inefficiency
- Before patents, technology progress was hindered by trade secrets
\subsubsection*{Patent Infringements}
- Must be found out by patent holder
- If court rules in favor of the patentee, patentee is entitled to injunction (infringer must stop using the patented technology) and damages (infringer must pay damages)
- Legal suits are difficult and expensive, therefore settlement or enhancement of negotitiation position are also options
- Patent trolls
\subsection*{Registered Designs}
- Design must be new
- Have individual character
- Registering period: 3-4 Months
- First five years 350€
- Can be registered 12 months after market introduction
- 25 years in total, renewal fees every 5 years
- Infringement enforcement can be difficult if not directly copied design
- Cheap but not very strong protection
\subsection*{Trademarks}
- Sign which can distinguish goods or services from on trader to another
- Can be words, logo, pictures
- Germany: 290€ fee for 10 years, can be renewed every 10 years indefinitly
- Must be used in order to maintain protection
\subsection*{Copyrights}
- Unregistered right, arises automatically
- Artistic works, or computer software
- In contrast to patents, protects the expression of ideas, not the idea itself
- 70 years in Germany, US even longer
\subsection*{IP Protection in Europe}
European Patent Convention (EPC), 1973, created "European Patent".
European Patent:
- Granted by European Patent Office
- After grant, group of independent national patents
- To come: EU-wide "Unitary Patent"
European Union Intellectual Property Office (EUIPO)
- European Union Trade Mark
- Registered Community Design
Cost of patenting in Europe:
- Filing fee: 125€ online, 260€ paper
- Search fee: 1,350€
- "page fee": 16€
- Examination fee: 1700€
- Designation Fee: 610€
- Fee for grant and printing: 960€
- Renewal fee: 490€ to 1640€
- Total cost: roughly 6000€, 3000€ for attorney

\section*{Designing the Innovation Process}
\subsection*{Innovation Funnel}
PSI = Product and Service Innovation
Phases:
- Capabilities/Market assessment and forecasting
- Development of goals and objectives
- Project portfolio planning
- Project management and execution
- Post-project learning and improvement
Problems in Reality:
- Unclear where to source ideas from
- Product filter disadvantages radical innovations
- Pet projects of senior management
- Products get jammed up and recirculate
- A lot of hot air at the end, but only few products make it to market
\subsection*{Open Innovation}
Strategy that encourages organizations to collaborate with external entities like customers, suppliers and even competitors to foster innovation.
\subsubsection*{R\&D Interactions with the Environment}
Inbound:
- Technology purchasing
- Acquisition
- Joint Ventures
- Exchange with Universities
- Ideas from Users
Outbound:
- Technology selling
- Creation of innovative firms
- Joint ventures
- Divestment
C\&D:
- Connect and Develop
- Create connections instead of own research
\subsection*{Stage-Gate Process}
- Designed around achieving key goals and success factors defined for PSI (Process and Service Innovation)
- Stages and gates break innovation into defined stages
- All work on a stage needs to be finished before passing a gate and entering a new stage
Stages and Gates:
- Stage 0: Discovery
- Gate 1: Idea screen
- Stage 1: Scoping
- Gate 2: Second screen
- Stage 2: Build business case
- Gate 3: Go to development
- Stage 3: Development
- Gate 4: Go to testing
- Stage 4: Testing \& Validation
- Gate 5: Go to Launch
- Stage 5: Launch
Post Launch review afterwards. Stage 2 (Build business case) is also called the "key homework stage".
\subsubsection*{Pros and Cons}
Good:
- Good structure for decision making on multiple levels
Bad:
- Process can be politically hijacked
- Choice of criteria determines throughput
- Process can be used as a Tyranny
- Does not allow incremental improvements
- Right incentives are critical
\subsection*{Agile Project Management}
- Accept unavoidable planning inaccuracies and unpredictable events
- Integrate backlogs with short sprints
- Learn and adjust by integrating customer feedback
- Promote collaboration
\subsection*{Common Failures of PSI Process}
The result...
- does not meet technical specs
- competes with other products in portfolio
- lacks strategic alignment
- does not meet user needs
- is too highly priced
- is too late to market
- is not sufficiently differentiated
- does not comply with regulation
\subsection*{Measurable Dimension of PSI performance}
- Productivity
- Speed to market
- Flexibility
- Quality
- Overall fit

\section*{Organizing R\&D and Innovation}
It is difficult to integrate R\&D with other business sections. Many firms face reduced R\&D productivity.
\subsection*{R\&D Production Interface}
Symptoms:
- Development cycles are long and expensive
- Late and unplanned changes in product design
Reasons:
- Development is too sequential
- High uncertainty when defining cost or deadlines
- Too many silos in companies, no information flow between departments
- Unclear strategic objectives of top management
Solutions:
- Overlapping development phases
- Matrix organisation (product x business function)
- Cross-functional teams (team composed of people from different functions)
- Proximity of relevant actors
- Use of suitable communication architecture
\subsection*{R\&D Marketing Interface}
- Empirical Studies confirm high importance of R\&D/Marketing interface for innovation
- Many conflicts between marketing and engineering personnel
Reasons:
- Difference in culture
- Different objectives
- Lack of trust and credibility in/of information from other function
Solutions:
- Split large projects in sub projects
- Early integration of both functions into innovation process
- Clear definition of competencies
- Open discussions
- Support contacts between individuals
- Integrating task force of management
- Cross functional development teams

\section*{Individuals in R\&D}
\subsection*{Resistance to Innovation}
Rational-technological arguments:
- "Does it work?"
- "Now is not a good time"
- "It does not fit to our processes/products"
Rational-economic arguments:
- "Existing capital goods will become obsolete"
- "The risk is too high"
- "Who on earth needs this?"
\subsection*{Causes of Resistance}
Barriers of "not-knowing":
- Innovation requires intensive learning
- Leads to feeling of inability to cope with innovation
- These individual effects get ampflified in groups/departments
Barriers of "not-wanting":
- Learned mechanisms of regulation
- Reasons related to power distribution
- Ideological world views, conservative attitude
- Objective reasons since innovation creeates winners and losers
Resistance from administration/accounting:
- Nobody feels responsible
- Pass idea up in the hierarchy
- Problems of coordination from ineffective assumption or allocation of competencies
- Filtering when innovation criticizes the existing organization
- Accounting does not treat innovation as assets or investments but as expenses
- According to IAS: Development must be amortized, research must not
- Accounting does not discount for future revenue
\subsection*{The "Promoter Model"}
- Developed out of research project "Columbus" from Witte (1973)
- Analyzed the decision process of adapting computers in companies
- Result: Strong forces existed against, but also for (promoters)
- Theory: Resistance against innovation requires power to overcome them
- Works best if different individuals take on different promoter roles
\subsubsection*{Types of Promoters}
Technical promoter:
- Has ideas for new product/process/technology
- Contributes creative effort
- Specific knowledge
- High technical credebility
- Reowned expert
- Known possibilities and limitations
Process promoter:
- Recognizes value
- Can identify relevant resources
- Involves other promoters/key individuals
- knows organization
- speaks language of both Management and R\&D
- Diplomatic/negotiator/Communicator/Charismatic
- Willing to take risks
Power promoter:
- Allocates resources
- Has hierarchical potential to overcome resistance
- Can allocate staff and financial resources
- Able to advance decision process
- Takes company strategy into account
- Long-term perspective
Guideline:
- Deficiencies of technical promoter need to be identified and compensated by power promoter
- Process promoter is needed when "large distances" are the case
- Self-organization is needed, individuals need to autonomously form a team
\subsection*{Networks and Gatekeepers}
The Broker:
- Connects clusters of disconnected users
The technological gatekeeper:
- Connects clusters of disconnected users
- Strongly connected internally and externally
- Gather and understand external information, translate information into something useful for organization
- Central person
- Good knowledge of external information sources
- Information collector/producer/catalyst
- Unlike promoter, not linked to specific project
- High technical competence
- Publications/presentation
- High formal education
- Experienced/employed for a long time
- Lower leadership level in hierarchy
- Role cannot simply be filled, existing gatekeeper needs to be identified and supported

\section*{Selected Tools and their Applications}
Innovation strategy requires up-to-date information
- Technological state state of the art
- Customer needs
- Competitor actions
- Market/demand analysis
\subsection*{Monitoring}
1. Check inventory of current knowledge
2. Planning of monitoring activities
3. Data collection
4. Reports
\subsubsection*{Sources of Information}
- Research institutions
- Suppliers
- Trade associations
- Other companies
- Competitors
- Customers
\subsubsection*{Types of Information}
- Activities in academic research
- Characteristics and applications of new technologies
- Technology-related information on institutions and companies
- Trends related to business and market aspects
- Industry news
- Public policy and legal framework (standards, regulations)
\subsubsection*{Methods of Data Collection}
- Field observation (reverse engineering, testing, visiting factories)
- Direct contact to experts
- Indirect contact to experts (consultants, licensing)
- Publications
- Data bases (patents, business data)
- Organizations
\subsection*{Roadmapping}
Process that contributes to the integration of business and technology and to the definition of technology strategy by displaying the interaction between products and technologies over time.
Can be displayed in two 2d charts with these axis: product/time and technology/time.
Objects on these graphs can be connected by arrows.
Is useful since technology and products interact in various ways:
- new tech $=>$ product improvement
- new tech $=>$ new product
- feedback about product $=>$ new tech
\subsubsection*{Benefits of Roadmapping}
- Establishment of shared product-technology strategy
- Interlinked approach for long-range product and technology planning
- Stimulation of learning and improvement of cross functional communication
- Improvement of time-to-market and time-to-money
\subsection*{Market Research: The Approach of User Integration}
What does the firm actually want?
1. Marketing:
- Increase people's brand/product awareness, identification
- Increase people's interaction with brand/products
2. Innovation: access and transfer external knowledge
- about problems: what are the needs they have
- about solutions: can they solve the problems you have?
\subsection*{Sticky Information}
- Information on needs and solutions is often sticky (costly to acquire or transfer)
- Users and manufacturers use local information when they innovate
- Richest need information is usually found at user sites
- Richest solution information is often found at manufacturer sites
This means innovation often comes from users:
- First new type of scientific instrument
- Conveyer belt in manufacturing (Ford was a user)
- First sports nutrition bar
- Footstraps for windsurf boards
Manufacturers tend to develop improvements. Manufacturers use this insight to move parts of their innovation process to where the sticky knowledge resides.
\subsection*{Lead Users}
1. Have needs that foreshadow general demand
2. Expect to obtain high benefit from a solution to their needs
3. Often innovate to solve their own needs by developing prototypes
4. Are often willing to freely reveal their innovations or start companies
- Lead user may be a firm or an individual
- Lead user innovation account for 10 to 40\% of innovation in many industries
- Often not customers of your firm
- They may not have an incentive to lead you to their innovations
- Manufacturer needs to identify lead user innovations (for example via lead user studies)
\subsubsection*{Lead User Method}
- This means innovation often comes from users
- Classical market research is based around concept of representative customer
- Lead User Method focuses on users that are ahead of the trend
- Lead users are not leading-edge customers (= customers trying new product first)
\subsection*{Double Use}
Innovation = new combination of need and solution.
Double Use = exisiting solution combined with new need.

\section*{Afuah A. - Models of Innovation}
\subsection*{What is Innovation?}
Innovation = invention + commercialization. The use of new technological or market knowledge to offer a product or service that customers want.
Two broad differentiations:
- Technical vs administratie innovations
- Product vs process innovations
\subsection*{Static models}
\subsubsection*{Incremental vs Radical Dichotomy}
- Organizational lens: does the new knowledge build on or destroy existing competences?
- Economic lens: does the new product render existing products non-competitive?
- Strategic view: incumbents favor incremental change, entrants favour radical change
- Organizational view: even with incentives, incumbents may lack the required capabilities
\subsubsection*{Abernathy-Clark model}
There is a difference between technological and market knowledge.
Four innovation types:
- Regular (both preserved)
- Niche (technical knowledge preserved, market knowledge destroyed)
- Revolutionary (tech. destroyed, market preserved)
- Architectural (both destroyed)
Incumbents can survive radical technological shifts if they own valuable market capabilities.
\subsubsection*{Henderson-Clark Architectural Model}
Separates component knowledge (knowledge about core components) from architectural knowledge (knowledge how these are linked together).
Four types:
- Incremental (both preserved)
- Modular (component knowledge destroyed, architectural knowledge preserved)
- Architectural (comp. preserved, arch. destroyed)
- Radical (both destroyed)
Explains why small tweaks in components can still trip established firms when linkages change.
\subsubsection*{Christensen's Disruptive Technology Model}
Disruptive innovation start in niche markets and continuous improvement makes them viable in mainstream markets.
Incumbents are too focused on their customers.
\subsubsection*{Innovation Value-Added Chain}
- An innovation that is incremental for the producer can be radical for suppliers, customers or complementary innovators
- Success depends on the whole ecosystem, not just the focal firms
\subsubsection*{Strategic-leadership view}
Adoption hinges on top managements dominant logic. Ability to recognize and resource innovation precedes incentives and capabilities.
\subsubsection*{Roberts-Berry familiartiy matrix}
Selecting entry mechanisms depends on how familiar the firm is with both the new technology and the target market.
\subsubsection*{Quantity \& Quality of Knowledge}
- High-tech, knowledge-based products show increasing returns, network effects and customer lock-in; bulk-processing products does not
- Tacit vs explicit knowledge matters
\subsubsection*{Teece's appropriability \& complementary-asset model}
- Profits flow to whoever controls scarce complementary assets when the core tech is easy to imitate, and to the innovator when its hard to imitate
\subsubsection*{Environment \& Strategic Choice}
- Local factor conditions, demanding customers, supportive suppliers and intense rivalry (Porters diamond) foster innovations






\end{document}
